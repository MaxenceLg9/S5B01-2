\documentclass{report}
\usepackage[T1]{fontenc}
\usepackage{graphicx}
\usepackage[table]{xcolor}
\usepackage[french]{babel}
\usepackage{titlesec}
\usepackage[a4paper]{geometry}
\usepackage{listings}
\usepackage[utf8]{inputenc}
\usepackage{lmodern}
\usepackage{babel}
\usepackage{stix}
\usepackage{minted}
\usepackage{fontspec}
\usepackage{tcolorbox}
\usepackage{hyperref}
\usepackage{titling}
\usepackage{enumitem}
\usepackage{fancyvrb}
\usepackage{tikz}
\usepackage{changepage}
\usepackage{tabularx}
\usepackage{float}
\usepackage{amsmath, amssymb}

\setmainfont{Calibri}



\setlist[itemize]{label=\large\textbullet}


\definecolor{azure}{rgb}{0.2, 0.7, 1.0}
\definecolor{bggray}{gray}{0.95}

\setlength{\parindent}{0pt}

\hypersetup{
	colorlinks=true,
	linkcolor=purple,
	filecolor=magenta,      
	urlcolor=blue,
	pdfborder={0 0 1}
}

\titleformat{\chapter}[block]
{\normalfont\LARGE\bfseries} % Style: large bold text
{\thechapter}                % Keep chapter number (remove if unwanted)
{1em}                        % Spacing between number and title
{}     

\urlstyle{same}

\geometry{width=18cm}
\geometry{a4paper}

\lstset{
	basicstyle=\ttfamily\small, % typewriter font
	keywordstyle=\color{blue}\bfseries, % keywords
	commentstyle=\color{green!50!black}\itshape, % comments
	stringstyle=\color{red}, % strings
	showstringspaces=false,
	numbers=none, % line numbers on the left
	numberstyle=\tiny\color{gray},
	backgroundcolor=\color{bggray},
	breaklines=true,
	frame=none,
	tabsize=4
}

\lstdefinelanguage{Rust}{
    keywords={fn, let, mut, if, else, match, impl, struct, enum, use, pub},
    sensitive=true,
    comment=[l]{//},
    morecomment=[s]{/*}{*/},
    morestring=[b]",
    basicstyle=\ttfamily\small,
    keywordstyle=\color{blue},
    commentstyle=\color{gray},
    stringstyle=\color{red},
}

\newenvironment{terminal}[1]{%
	\Verbatim[frame=none, numbers=none,label={#1}, breaklines, breakanywhere,tabsize=4,breaksymbol=, breakanywheresymbolpre=,backgroundcolor=bggray]%
}{%
	\endVerbatim
}

\renewcommand{\thechapter}{\Roman{chapter}}
\renewcommand{\thesection}{\thechapter.\Alph{section}}
\renewcommand{\thesubsection}{\thesection.\arabic{subsection}}
\renewcommand{\thesubsubsection}{\thesubsection.\alph{subsubsection}}


\pretitle{%
	\begin{center}
		\LARGE
		\includegraphics[width=6cm,height=2cm]{../../../../../../Format/logo-UT-site.png}\\[\bigskipamount]
	}
	\posttitle{\end{center}}

\setcounter{secnumdepth}{4}
\setcounter{tocdepth}{3} 

\title{\Huge{\bfseries S5.B.01 Phase 4\\Déploiement de services}}
\date{\today}
\author{Maxence Lagourgue}

\begin{document}
	
	\maketitle
	\tableofcontents
	
	\chapter{Introduction}
	
	\section{Outils}
	
	Dans cette partie, les outils utilisés seront:
	\begin{itemize}
	\item Rancher pour la gestion des clusters
	\item RKE2 pour la mise en œuvre Kubernetes des nœuds de travail
	\item k3s pour le cluster Rancher
	\item kubectl pour la gestion des ressources
	\item Helm pour la gestion des applications
	\end{itemize}
	
	Plus tard, si nous avons le temps, nous utiliserons Ansible pour automiser la chaîne de production Rancher.
	
	\section{Machines}
	
	Les machines utilisées au cours de ce projet seront:
	
	\begin{itemize}
	\item applicatif $\Longrightarrow$ k3s cluster + Rancher Server
	\item K8SA2 (k8s1) $\Longrightarrow$ RKE2 cluster + Master, Etcd, Worker Nodes
	\item K8SB2 (k8s2) $\Longrightarrow$ RKE2 cluster + Worker nodes + Backup
	\item K8SC2 (k8s3) $\Longrightarrow$ ???
	\end{itemize}
	
	\section{Configuration générale}
	
	Dans toutes les VMs impliquées dans le cluster Kubernetes, la configuration suivante sera définie.
	
	\begin{lstlisting}[language=Bash,caption={/etc/hosts}]
	10.0.1.3        rancher.rancher
	10.0.1.4        master.rancher
	10.0.1.5        worker.rancher
	\end{lstlisting}
	
	Pour monitorer les ressources des VMs, nous ne pouvons pas nous servir des indications données par Proxmox VM car le Guest Agent est désactivé. Nous aurons donc de mauvaises indications pour la RAM par exemple. Je conseille donc:
	
	\begin{lstlisting}[language=Bash,caption={}]
	wget https://github.com/fastfetch-cli/fastfetch/releases/download/2.56.1/fastfetch-linux-amd64.deb && dpkg -i fastfetch-linux-amd64.deb
	\end{lstlisting}
	

	\chapter{Installation de Rancher dans un cluster k3s}
	
	Pour utiliser Rancher, plusieurs méthodes d'installation s'offrent à nous.
	L'une avec docker, l'autre en tant que noeud Kubernetes. 
	Les autres installations reposent sur l'utilisation d'un Cloud Provider ainsi que Terraform donc inutile dans notre cas.
	
	Exemple de tutorial: \href{https://blog.stephane-robert.info/docs/conteneurs/orchestrateurs/outils/rancher/}{Tutorial Rancher 2025}
	
	Faire le gitlab en tant qu'application kubernetes/rancher.
	
	
	\section{Installation de k3s}
	
	\begin{terminal}{Installation de k3s}
	curl -sfL https://get.k3s.io | INSTALL_K3S_VERSION="v1.24.10+k3s1" sh -s - server --cluster-init --bind-address 10.0.1.3
	\end{terminal}
	
	\subsection{Accès distant au cluster (Optionnel)}
	
	\verb*|<IP_OF_LINUX_MACHINE>| est l'IP de la machine distante sur laquelle se trouve le cluster.
	
	\begin{terminal}{}
	scp root@<IP_OF_LINUX_MACHINE>:/etc/rancher/k3s/k3s.yaml ~/.kube/config
	\end{terminal}

	\begin{terminal}{Modifier l'URL du serveur rancher}
	nano ~/.kube/config
	\end{terminal}
	
	\subsection{Installation d'helm}
	
	\begin{terminal}{}
	sudo apt-get install curl gpg apt-transport-https --yes
	
	curl -fsSL https://packages.buildkite.com/helm-linux/helm-debian/gpgkey | gpg --dearmor | sudo tee /usr/share/keyrings/helm.gpg > /dev/null
	
	echo "deb [signed-by=/usr/share/keyrings/helm.gpg] https://packages.buildkite.com/helm-linux/helm-debian/any/ any main" | sudo tee /etc/apt/sources.list.d/helm-stable-debian.list
	
	sudo apt-get update
	sudo apt-get install helm
	\end{terminal}
	
	\subsection{Installation de kubectl}
	
	\begin{terminal}{}
	sudo apt-get install -y apt-transport-https ca-certificates curl gnupg
	
	curl -fsSL https://pkgs.k8s.io/core:/stable:/v1.34/deb/Release.key | sudo gpg --dearmor -o /etc/apt/keyrings/kubernetes-apt-keyring.gpg
	
	sudo chmod 644 /etc/apt/keyrings/kubernetes-apt-keyring.gpg
	
	echo 'deb [signed-by=/etc/apt/keyrings/kubernetes-apt-keyring.gpg] https://pkgs.k8s.io/core:/stable:/v1.34/deb/ /' | sudo tee /etc/apt/sources.list.d/kubernetes.list
	sudo chmod 644 /etc/apt/sources.list.d/kubernetes.list
	
	sudo apt-get update
	sudo apt-get install -y kubectl
	\end{terminal}
	
	\subsection{Création d'un Déploiement Rancher}
	
	<Hostname> correspond au nom de domaine utilisé pour contacter le pod rancher.
	
	\begin{terminal}{Creation du pod Rancher avec Helm}
	export KUBECONFIG=/etc/rancher/k3s/k3s.yaml
	

	
	kubectl create namespace cattle-system 
	kubectl config set-context --current --namespace=cattle-system 
	
	kubectl apply -f https://github.com/cert-manager/cert-manager/releases/download/v1.19.2/cert-manager.crds.yaml
	
	helm repo add rancher-latest https://releases.rancher.com/server-charts/latest
	
	helm repo add jetstack https://charts.jetstack.io
	
	helm repo update
	
	helm install cert-manager jetstack/cert-manager \
	  --namespace cert-manager \
	  --create-namespace
	  
	helm install rancher rancher-latest/rancher \
	  --namespace cattle-system \
	  --set hostname=rancher.rancher \
	  --set replicas=1 \
	  --set bootstrapPassword=testpassword
	\end{terminal}
	
	Il faut maintenant attendre car l'installation nécessite quelques minutes.
	On peut vérifier avec \verb*|kubectl get pods -n cattle-system|
	
	Une fois fait, on se connecte à la page et on récupère le mot de passe:
	\begin{terminal}{}
	kubectl get secret --namespace cattle-system bootstrap-secret -o go-template='{{.data.bootstrapPassword|base64decode}}{{"\n"}}'
	\end{terminal}
	
	On définit un nouveau mot de passe qui est \verb*|qJHiA@wwaagi46U|.
	
	\subsection{Pour arrêter/pauser Rancher}
	
	\begin{terminal}{}
	kubectl scale --replicas=0 deployment/rancher -n cattle-system
	\end{terminal}
	
	Cela permet d'arrêter temporairement le pod Rancher.
	
	\section{Réinitialisation du cluster}
	
	Pour revenir à l'état 0 du cluster, il est possible de:
	\begin{terminal}{}
	rm -rf /var/lib/rancher/k3s/server/db/etcd
	/usr/local/bin/k3s-killall.sh
	systemctl restart k3s.service
	\end{terminal}
	
	\chapter{Création d'un cluster}
	
	\section{Cluster pour l'application Nailloux}
	
	\section{Cluster pour les composants divers, Git, DockerRegistry...}
	
	\subsubsection{Enregistrement}
	
	\begin{lstlisting}[language=Bash,caption={}]
	curl --insecure -fL https://rancher.rancher/system-agent-install.sh | sudo  sh -s - --server https://rancher.rancher --label 'cattle.io/os=linux' --token 272xjzq28nm5xp4cpjjrxt4zqrwvsmftq45xs5bx4vv7kk65mh72pv --ca-checksum fdc9c50ea58442994213e96883b6a5ca39227fc7d4116e60fa1026c123f56583 --etcd --controlplane --worker --address 10.0.1.6 --internal-address 10.0.1.6
	\end{lstlisting}
	
	\section{Enrôlement de machines}
	
	Pour cela il faut aller dans la section \textbf{Clusters $\Longrightarrow$ <Cluster> $\Longrightarrow$ Registration}
	
	\begin{lstlisting}[language=Bash,caption={Machine master}]
	export CRI_CONFIG_FILE=/var/lib/rancher/rke2/agent/etc/crictl.yaml
	export CONTAINERD_ADDRESS=unix:///run/k3s/containerd/containerd.sock
	export PATH=$PATH:/var/lib/rancher/rke2/bin
	export KUBECONFIG=/etc/rancher/rke2/rke2.yaml
	
	curl --insecure -fL https://rancher.rancher/system-agent-install.sh | sudo  sh -s - --server https://rancher.rancher --label 'cattle.io/os=linux' --token tvg69x5vkm9szzlzsj6qqkx7wggzrgvt2grc755nth29h2ncjbgthz --ca-checksum fdc9c50ea58442994213e96883b6a5ca39227fc7d4116e60fa1026c123f56583 --etcd --controlplane --worker --address 10.0.1.4 --internal-address 10.0.1.4
	\end{lstlisting}
	
	\begin{lstlisting}[language=Bash,caption={Machine worker}]
	export CRI_CONFIG_FILE=/var/lib/rancher/rke2/agent/etc/crictl.yaml
	export CONTAINERD_ADDRESS=unix:///run/k3s/containerd/containerd.sock
	export PATH=$PATH:/var/lib/rancher/rke2/bin
	export KUBECONFIG=/etc/rancher/rke2/rke2.yaml
	
	curl --insecure -fL https://rancher.rancher/system-agent-install.sh | sudo  sh -s - --server https://rancher.rancher --label 'cattle.io/os=linux' --token tvg69x5vkm9szzlzsj6qqkx7wggzrgvt2grc755nth29h2ncjbgthz --ca-checksum fdc9c50ea58442994213e96883b6a5ca39227fc7d4116e60fa1026c123f56583 --worker --address 10.0.1.5 --internal-address 10.0.1.5
	\end{lstlisting}
	
	Si DNS issue:
	
	\begin{lstlisting}[language=Bash,caption={}]
	export KUBE_EDITOR="nano"
	kubectl edit configmap rke2-coredns-rke2-coredns -n kube-system
	
	hosts {
		10.0.1.3 rancher.rancher
		fallthrough
	}
	
	kubectl rollout restart deployment rke2-coredns-rke2-coredns -n kube-system
	\end{lstlisting}
	
	\chapter{Creation d'un Workload}
	
	Tout d'abord, avant de déployer quoi que ce soit il faut convertir le docker-compose.yml en fichier de déploiement Kubernetes, un manifest.
	

	
\end{document}