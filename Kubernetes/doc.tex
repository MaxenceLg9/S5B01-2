\documentclass{report}
\usepackage[T1]{fontenc}
\usepackage{graphicx}
\usepackage[table]{xcolor}
\usepackage[french]{babel}
\usepackage{titlesec}
\usepackage[a4paper]{geometry}
\usepackage{listings}
\usepackage[utf8]{inputenc}
\usepackage{lmodern}
\usepackage{babel}
\usepackage{minted}
\usepackage{fontspec}
\usepackage{tcolorbox}
\usepackage{hyperref}
\usepackage{titling}
\usepackage{enumitem}
\usepackage{fancyvrb}
\usepackage{tikz}
\usepackage{changepage}
\usepackage{tabularx}
\usepackage{float}
\usepackage{amsmath, amssymb}

\setmainfont{Calibri}



\setlist[itemize]{label=\large\textbullet}


\definecolor{azure}{rgb}{0.2, 0.7, 1.0}
\definecolor{bggray}{gray}{0.95}

\setlength{\parindent}{0pt}

\hypersetup{
	colorlinks=true,
	linkcolor=purple,
	filecolor=magenta,      
	urlcolor=blue,
	pdfborder={0 0 1}
}

\titleformat{\chapter}[block]
{\normalfont\LARGE\bfseries} % Style: large bold text
{\thechapter}                % Keep chapter number (remove if unwanted)
{1em}                        % Spacing between number and title
{}     

\urlstyle{same}

\geometry{width=18cm}
\geometry{a4paper}

\lstset{
	basicstyle=\ttfamily\small, % typewriter font
	keywordstyle=\color{blue}\bfseries, % keywords
	commentstyle=\color{green!50!black}\itshape, % comments
	stringstyle=\color{red}, % strings
	showstringspaces=false,
	numbers=none, % line numbers on the left
	backgroundcolor=\color{bggray},
	breaklines=true,
	frame=none,
	tabsize=4
}

\lstdefinelanguage{Rust}{
    keywords={fn, let, mut, if, else, match, impl, struct, enum, use, pub},
    sensitive=true,
    comment=[l]{//},
    morecomment=[s]{/*}{*/},
    morestring=[b]",
    basicstyle=\ttfamily\small,
    keywordstyle=\color{blue},
    commentstyle=\color{gray},
    stringstyle=\color{red},
}

% Define a new environment "varblock" with 1 argument = label
\newenvironment{terminal}[1]{%
	\Verbatim[frame=none, numbers=none,label={#1}, breaklines, breakanywhere,tabsize=4,breaksymbol=, breakanywheresymbolpre=,backgroundcolor=bggray]%
}{%
	\endVerbatim
}


\renewcommand{\thechapter}{\Roman{chapter}}
\renewcommand{\thesection}{\thechapter.\Alph{section}}
\renewcommand{\thesubsection}{\thesection.\arabic{subsection}}
\renewcommand{\thesubsubsection}{\thesubsection.\alph{subsubsection}}


\pretitle{%
	\begin{center}
		\LARGE
		\includegraphics[width=6cm,height=2cm]{../../../../../../Format/logo-UT-site.png}\\[\bigskipamount]
	}
	\posttitle{\end{center}}

\setcounter{secnumdepth}{4}
\setcounter{tocdepth}{3} 

\title{\Huge{\bfseries S5.B.01 Phase 4\\Déploiement de services}}
\date{\today}
\author{Maxence Lagourgue}

\begin{document}
	
	\maketitle
	\tableofcontents
	
	\chapter{Outils}
	
	Dans cette partie, les outils utilisés seront:
	\begin{itemize}
	\item Rancher pour la gestion des clusters
	\item RKE2 pour la mise en œuvre Kubernetes des nœuds de travail
	\item k3s pour le cluster Rancher
	\item kubectl pour la gestion des ressources
	\item Helm pour la gestion des applications
	\end{itemize}

	\chapter{Création du cluster avec Rancher}
	
	Pour utiliser Rancher, plusieurs méthodes d'installation s'offrent à nous.
	L'une avec docker, l'autre en tant que noeud Kubernetes. 
	Les 
	
	\section{Installation de rancher}
	
	\begin{terminal}{Installation de k3s}
	curl -sfL https://get.k3s.io | INSTALL_K3S_VERSION="v1.24.10+k3s1" sh -s - server --cluster-init
	
	# Optional, to access to the cluster remotely
	scp root@<IP_OF_LINUX_MACHINE>:/etc/rancher/k3s/k3s.yaml ~/.kube/config
	\end{terminal}

	\begin{terminal}{Modifier l'URL du serveur rancher}
	nano ~/.kube/config
	\end{terminal}
	
	\begin{terminal}{Creation du pod Rancher avec Helm}
	helm repo add rancher-latest https://releases.rancher.com/server-charts/latest
	
	kubectl create namespace cattle-system
	
	kubectl apply -f https://github.com/cert-manager/cert-manager/releases/download/<VERSION>/cert-manager.crds.yaml
	
	helm repo add jetstack https://charts.jetstack.io
	
	helm repo update
	
	helm install cert-manager jetstack/cert-manager \
	  --namespace cert-manager \
	  --create-namespace
	  
	helm install rancher rancher-latest/rancher \
	  --namespace cattle-system \
	  --set hostname=<IP_OF_LINUX_NODE>.sslip.io \
	  --set replicas=1 \
	  --set bootstrapPassword=<PASSWORD_FOR_RANCHER_ADMIN>
	\end{terminal}
	
	
	
	Utilisation de Helm, Kubectl, k3s, RKE2.
	
	Faire le gitlab en tant qu'application kubernetes/rancher.
	
\end{document}
