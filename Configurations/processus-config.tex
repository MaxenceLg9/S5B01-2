\documentclass{report}
\usepackage[T1]{fontenc}
\usepackage{graphicx}
\usepackage[table]{xcolor}
\usepackage[french]{babel}
\usepackage{titlesec}
\usepackage[a4paper]{geometry}
\usepackage{listings}
\usepackage[utf8]{inputenc}
\usepackage{lmodern}
\usepackage{babel}
\usepackage{stix}
\usepackage{minted}
\usepackage{fontspec}
\usepackage{tcolorbox}
\usepackage{hyperref}
\usepackage{titling}
\usepackage{enumitem}
\usepackage{fancyvrb}
\usepackage{tikz}
\usepackage{changepage}
\usepackage{tabularx}
\usepackage{float}
\usepackage{amsmath, amssymb}

\setmainfont{Calibri}



\setlist[itemize]{label=\large\textbullet}


\definecolor{azure}{rgb}{0.2, 0.7, 1.0}
\definecolor{bggray}{gray}{0.95}

\setlength{\parindent}{0pt}

\hypersetup{
	colorlinks=true,
	linkcolor=purple,
	filecolor=magenta,      
	urlcolor=blue,
	pdfborder={0 0 1}
}

\titleformat{\chapter}[block]
{\normalfont\LARGE\bfseries} % Style: large bold text
{\thechapter}                % Keep chapter number (remove if unwanted)
{1em}                        % Spacing between number and title
{}     

\urlstyle{same}

\geometry{width=18cm}
\geometry{a4paper}

\lstset{
	basicstyle=\ttfamily\small, % typewriter font
	keywordstyle=\color{blue}\bfseries, % keywords
	commentstyle=\color{green!50!black}\itshape, % comments
	stringstyle=\color{red}, % strings
	showstringspaces=false,
	numbers=none, % line numbers on the left
	numberstyle=\tiny\color{gray},
	backgroundcolor=\color{bggray},
	breaklines=true,
	frame=none,
	tabsize=4
}

\lstdefinelanguage{Rust}{
    keywords={fn, let, mut, if, else, match, impl, struct, enum, use, pub},
    sensitive=true,
    comment=[l]{//},
    morecomment=[s]{/*}{*/},
    morestring=[b]",
    basicstyle=\ttfamily\small,
    keywordstyle=\color{blue},
    commentstyle=\color{gray},
    stringstyle=\color{red},
}

\newenvironment{terminal}[1]{%
	\Verbatim[frame=none, numbers=none,label={#1}, breaklines, breakanywhere,tabsize=4,breaksymbol=, breakanywheresymbolpre=,backgroundcolor=bggray]%
}{%
	\endVerbatim
}

\renewcommand{\thechapter}{\Roman{chapter}}
\renewcommand{\thesection}{\thechapter.\Alph{section}}
\renewcommand{\thesubsection}{\thesection.\arabic{subsection}}
\renewcommand{\thesubsubsection}{\thesubsection.\alph{subsubsection}}


\pretitle{%
	\begin{center}
		\LARGE
		\includegraphics[width=6cm,height=2cm]{../../../../../../Format/logo-UT-site.png}\\[\bigskipamount]
	}
	\posttitle{\end{center}}

\setcounter{secnumdepth}{4}
\setcounter{tocdepth}{3} 

\title{\Huge{\bfseries S5.B.01 Phase 4\\Configurations Réseaux - Processus}}
\date{\today}
\author{Maxence Lagourgue}

\begin{document}
	
	\maketitle
	\tableofcontents
	
	\chapter{Introduction}
	
	Ce document a pour vocation d'exposer tous les artéfacts créés / utilisés pour optimiser les processus de configurations dans le cadre de cette SAE.
	
	\chapter{VMs}
	
	Dans cette section, nous évoquerons les outils utilisés pour faciliter la définition des configurations des interfaces réseaux notamment, ainsi que le déploiement du Site Web et des Clusters Kubernetes (RKE2 / k3s).
	
	\section{Configurations Interfaces Réseaux}
	
	Pour cette partie, seront détayés les différentes étapes afin de récupérer et d'envoyer les fichiers *.network pour les interfaces réseaux au travers de ssh via sa sous-commande scp.
	
	\subsection{Création d'un utilisateur qui aura la propriété des fichiers}
	
	Pour plus de sécurité, nous n'utiliserons par l'utilisateur root avec une clé ssh mais un utilisateur spécial qui aura le droit de lecture seulement sur /etc/systemd/network, et sur tous les dossiers nécessitant un accès spécial dont nous voulons récupérer et définir la configuration.
	
	Cet utilisateur se nommera:  sisyphe.
	
	On réalise cette manipulation avec:
	\begin{lstlisting}[language=Bash,caption={}]
	useradd -m sisyphe
	passwd sisyphe
	chown sisyphe:sisyphe -R /etc/systemd/network/
	\end{lstlisting}
	
	Si comme moi, vous avez oublié l'option \textbf{-m} pour créer le répertoire 
	utilisateur dans /home, vous pouvez exécuter la commande suivante.
	
	\begin{lstlisting}[language=Bash,caption={}]
	install -d -m 0700 -o sisyphe -g sisyphe /home/sisyphe
	\end{lstlisting}
	
	\subsection{Définition d'une clé SSH }
	
	Maintenant, donc devons donc générer une clé SSH pour pouvoir par la suite activer la connexion sans mot de passe.
	
	\begin{lstlisting}[language=Bash,caption={}]
	ssh-keygen -f sisyphe_key
	\end{lstlisting}
	
	\subsection{Liaison de la clé SSH au compte sisyphe}
	
	\begin{lstlisting}[language=Bash,caption={}]
	ssh-copy-id -i sisyphe_key.pub sisyphe@10.0.1.3
	\end{lstlisting}
	
	\subsection{Fichier exécutable pour automiser le processus}
	
	Maintenant, nous allons créer deux fichiers, l'un pour récupérer les fichiers distants, l'un pour les envoyer.
	
	\begin{lstlisting}[language=Bash,caption={get-config}]
	#!/bin/bash
	
	scp -v -i ./sisyphe_key sisyphe@10.0.1.2:/etc/systemd/network/* Reseau/
	scp -i ./sisyphe_key sisyphe@10.0.1.3:/etc/systemd/network/* Applicatif/
	scp -i ./sisyphe_key sisyphe@10.0.1.4:/etc/systemd/network/* k8sa2/
	scp -i ./sisyphe_key sisyphe@10.0.1.5:/etc/systemd/network/* k8sb2/
	scp -i ./sisyphe_key sisyphe@10.0.1.6:/etc/systemd/network/* k8sc2/
	scp -i ./sisyphe_key sisyphe@10.0.1.10:/etc/systemd/network/* Backup/
	\end{lstlisting}
	
	\begin{lstlisting}[language=Bash,caption={push-config}]
	#!/bin/bash
	
	scp -i sisyphe_key Applicatif/* sisyphe@10.0.1.2:/etc/systemd/network/
	scp -i sisyphe_key Reseau/* sisyphe@10.0.1.3:/etc/systemd/network/
	scp -i sisyphe_key k8sa2/* sisyphe@10.0.1.4:/etc/systemd/network/ 
	scp -i sisyphe_key k8sb2/* sisyphe@10.0.1.5:/etc/systemd/network/
	scp -i sisyphe_key k8sc2/* sisyphe@10.0.1.6:/etc/systemd/network/
	scp -i sisyphe_key Backup/* sisyphe@10.0.1.10:/etc/systemd/network/
	\end{lstlisting}
	
	Il faut définir ces fichiers exécutables avec la commande suivante:
	\begin{lstlisting}[language=Bash,caption={}]
	chmod +x ./get-config ./push-config
	\end{lstlisting}
	
	\section{Configuration du fichier /etc/hosts}
	
	Pour cette configuration on va utiliser le même principe que précédemment sauf que le fichier est le même pour toutes les VMs donc copiera le même fichier à chaque fois.
	
	Il sera d'ailleurs seulement utile de le push.
	
	
	
\end{document}
